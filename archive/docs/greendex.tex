\documentclass[12pt]{article}
\usepackage[normalem]{ulem}
\setlength{\textheight}{9.5in}
\setlength{\topmargin}{-1in}

\setlength{\textwidth}{6.5in}
\setlength{\oddsidemargin}{0.0in}
\setlength{\evensidemargin}{0.0in}

\newcommand{\TODO}[1]{\textbf{TODO: } \emph{#1}}

\newenvironment{tightlist}%
{\begin{list}{$\bullet$}{%
    \setlength{\topsep}{0in}
    \setlength{\partopsep}{0in}
    \setlength{\itemsep}{0in}
    \setlength{\parsep}{0em}
    \setlength{\leftmargin}{1.5em}
    \setlength{\rightmargin}{0in}
    \setlength{\itemindent}{0in}
}
}%
{ \end{list} }

\begin{document}

\begin{center} \def\baselinestretch{1.25}
\huge
The Sacred and Solemn Rites of the \\
GREENDEX\\
\vfil
\Large
Being a description of the  \\
Operation of the Digital \\ 
Checkout System of the Library. \\
\vfil
\normalsize
\today
\end{center}

\thispagestyle{empty}

\newpage

\addtocounter{page}{-1}

\section{Getting Started}

Connect to Athena somehow (psi-phi, the computer on the Other Desk, may be
useful in this respect), and run \texttt{athrun mitsfs greendex} to start
Greendex.

You should be looking at a screen that looks like:

\begin{verbatim}
This is greendex 1.1

Main Menu

S. Select Patron/Member
I. Check In Books by Author or Barcode
N. New Patron
D. Display Book
A. Book Drop/Fancy Check In
Q. Quit
selection: 
\end{verbatim}

If you're not looking at such a screen (or if you later encounter
difficulties with Greendex that you cannot overcome otherwise), contact
OokComm (\texttt{ookcomm@mit.edu}) and the Star Chamber
(\texttt{star-chamber@mit.edu}) for help.  If you aren't able to resolve
the issue in a sufficiently timely manner, use the paper log
book\footnote{I'm not sure this exists, and it really should.}
located with the Money Book and Keyholder Notes to record any
transactions that occur while you're unable to use the Greendex.

As should be obvious, typing the first character of a line selects that
option.

If you have any bug reports or feature requests please contact
(\texttt{ookcomm@mit.edu})

\section{Overview}

The Greendex's menus form a maze of twisty passages, some alike.
Fortunately, most menus will allow you to return to the Main Menu in a
single keystroke.

The Start Menu contains all of the options available when no patron is
selected.  The menus not directly linked to from there require a patron
to be selected, and navigating from one of those menus to another will
retain that patron as selected.

These Rites do not exhaustively list all options provided in the menus,
but rather focus on providing information that may not be apparent
from the interface itself.

\section{General Information}

Control-C will cancel the current operation.

A single \texttt{?} on a line by itself is a prompt to type the number
of an option provided on the lines preceding.  (This may not always be
obvious when there's only a single option provided.)  If you don't want
to select any of the provided options, use Control-C to cancel.

When revisiting a menu, you can use the up and down arrow to cycle
through previously-made selections.

\section{Start Menu}

As mentioned earlier, this menu contains the options available when
no patron is selected.  This is also the menu from which you can quit the
application.

\section{New Patron}

This option creates a new patron.

Of particular note is the ``Nickname'' field, which permits an
additional name to be associated with the patron.  When selecting a patron
(see below), any nicknames will be included in the search.\footnote{Additional
nicknames can be added from the \texttt{Edit Patron and Membership} menu
under the \texttt{Main Menu}.}

It's important that the email field is valid.

While you don't have to enter an ID number into Greendex, you should
still check the patron's information against a photo ID.

You should have the patron fill out a membership and transfer it when
they are reading the rules. Have them sign it and file it away with
the other old sheets.

\section{Select Patron/Member}

Selecting a patron performs a search on patron names, and provides a
list of matches to select from.  Each set of letters separated by spaces
will be treated as a substring to search for, in order -- ``hin hrog''
will match to ``Phineas Phrogg'' or, for that matter, to ``Wendy
Shinthroge''.

Once you have selected a patron, you arrive at the \texttt{Main Menu}.

\section{Main Menu}

The \texttt{Main Menu} is the central navigation point in Greendex's menus, and
provides access to the bulk of the available functions.

At the top of this menu, there is a line of information provided
about that patron:

\begin{verbatim}
Member:  Phineas Phrogg <phrogg@mit.edu> $0.00
Membership: Year Expires: 2016-07-19
\end{verbatim}

The dollar value is their current balance with MITSFS -- if they owe
fines, it's negative; if they have credit, it's positive.

To renew a patron's membership, use the \texttt{E. Edit Patron and
  Membership} option -- do \textbf{not} use the \texttt{Membership} option
under \texttt{Financial Transactions}.

The \texttt{Unselect Patron} option will return you to the \texttt{Start
  Menu}.


\section{Checking Books In And Out}

\sout{Check for a barcode. If there isn't one on checkout, add one and scan
it.}  \emph{We're not using barcodes just yet.}

Searches for author and title are both start-of-field searches.
``Heinl'' matches ``Heinlein, Robert A.'', but ``Robert'' does not.  Tab
completion is supported as well.

Remember that a leading ``The '', ``A '', and ``An '' is put at the end of
the title field in PinkDex, so ``A Dance'' does \emph{not} match ``Dance
With Dragons, A''.

If you enter partial information, Greendex will do its best to find books
matching the prefixes you gave it, and give you a list to select from.  If
you get something that is unambiguous, it will go ahead and select it for
you\footnote{For instance, author
  ``\texttt{Z}'', title ``\texttt{NEV}'' uniquely identifies a pretty
  good book!}.

Greendex will automatically levy fines for overdue books that are
returned.

The Winter Time Warp is handled automatically by Greendex.  However, the
details of the Time Warp do not currently appear in the interface.

\subsection{Declare Books Lost}

Just select the book the patron has lost; an appropriate fine will be
assessed.  It will continue to appear on their checked out book list, but
not count against their eight checked out books.  If they then find or
replace it, just check it back in (if they replace it with a different
format, write a note to that effect and leave it in the book, and leave the
book on the Panthercomm shelf).

\subsection{Fancy Checkin/Checkout}

These options are used for checkins and checkouts that violate the
normal rules, for example:
\begin{tightlist}
\item Checking out a reserve copy of a book\footnote{We'll assume you have
  Skinnerial Authorization or a really good reason}.
\item Manually specifying the checkin fine to be other than would
  normally be assessed.
\item Specifying the checkin/checkout date to be other than the current
  date.
\end{tightlist}

The last above is particularly useful when checking in books in the
bookdrop.

Don't use these options for normal checkin or checkout.

\section{Financial Transactions}

All of the options in the \texttt{Financial Transactions} menu and its
submenus generally do the same thing -- they add to or subtract money
from the patron's balance.  In general, the different options are for
the different categories of credits and fees.

\texttt{Void Previous} selects an un-voided transaction to undo. It
should always be selected to undo a transaction.

Positive values are credits and added to the patron's balance; negative
sums are fees and subtracted from the patron's balance.  \textbf{Note:}
when crediting a patron with a positive dollar value, the UI will assert
that you are ``Charging'' the patron the money -- do not be alarmed, they
will actually end up with more money in their balance at the end of the
operation.

In general, \textbf{any actual transfer of money to or from the Library}
should use the Payment or Reimbursement options -- those are different
from the others in that they also are recorded against the Library's
current cash.


\section{View Patron}

This menu displays assorted account information and has options for
displaying the Checkout, Financial, and Membership histories of the
patron.

\section{Edit Patron and Membership}

This is the menu to use to renew a patron's membership, as well as to
update their account information.

Additional entries can be added to the patron's Names field. The
defaults are shown in most views.

\section{Rites}

These Rites are in \verb$/mit/mitsfs/rites/greendex.tex$ (also as .ps
and .pdf), which as a Keyholder (or Prentice) you can read.  There
should be hardcopy near the Keyholder Notes and Moneybook.


\appendix
\section{Menus}
\begin{verbatim}

Main Menu:
S. Select Patron
I. Check In Books
N. New Patron
D. Display Book
A. Book Drop/Fancy Check In
   B. Book Drop Check In
   I. Fancy Check In
   Q. Main Menu
Q. Quit

Patron Menu:
O. Check Out Books by Author or Barcode
I. Check In Books by Patron/Member
L. Declare Books Lost
V. View Patron
   C. Check Out History
   F. Financial History
   M. Membership History
   Q. Main Menu
E. Edit Patron and Membership
   M. New/Renew Membership
   A. Add Info
      N. Add Name
      E. Add Email
      A. Add Address
      Q. Back to Edit Membership
   R. Remove Info
      N. Remove Name
      E. Remove Email
      A. Remove Address
      Q. Back to Edit Membership
   D. Set Default Info
      N. Set Default Name
      E. Set Default Email
      A. Set Default Address
      Q. Back to Edit Membership
   Q. Main Menu
F. Financial Transaction
   D. Donation for Fine Credit
   F. Asssess Fine
   K. Asssess Keyfine
   P. Payment
   A. Advanced Transactions
      L. LHE
      M. Membership
      O. Other
      R. Reimbursement
      V. Void Previous
      Q. Back to Financial Transactions
   Q. Back to Main Menu
A. Advanced (Book Drop/Fancy Check In/Check Out)
   B. Book Drop Check In
   I. Fancy Check In
   O. Fancy Check Out
   Q. Main Menu
Q. Unselect Member

\end{verbatim}

\end{document}
